% SPDX-License-Identifier: CC-BY-SA-4.0
% Copyright (c) 2026 Chandradeep Dey
% Copyright (c) 2026 Jeremy Stratton-Smith

\documentclass{scrartcl}
\usepackage{graphicx} % Required for inserting images

\renewcommand{\thesection}{\Roman{section}} 

\title{The Constitution}
\author{Formal Methods \& Logic at Rochester Institute of Technology}
\date{}

\begin{document}

\maketitle

\section{NAME, PURPOSE, AND AFFILIATION}
§ 1. The name of this organization shall be Formal Methods \& Logic at Rochester Institute of Technology, also referred to as FM\(\lambda\)@RIT.

§ 2. The purpose of this organization is to promote engagement among students and researchers interested in formal methods, logic, and related areas such as programming languages, software engineering, and cybersecurity (hereinafter referred to as topics of interest) through reading groups, seminars, and networking opportunities.

§ 3. This organization is not affiliated with any local, state, or national organization.

\section{MEMBERSHIP COMPOSITION}
§ 1. Membership is open to all students in good academic standing as defined in the RIT Governance Policy §D05.1.II. Continued membership for students is subject to maintaining good academic standing.

Membership is open to all faculty and staff of RIT. Continued membership is subject to their continued employment at RIT.

§ 2. Members with mentors (defined in Section 3) have voting privileges as described in ARTICLE IV. Additionally, members attending at least half of the non-special events (described in ARTICLE VI) throughout an academic year shall have voting privileges that year, subject to them maintaining said attendance. All other members are nonvoting. 

§ 3. Mentors are members who are faculty or staff of RIT who supervise/have supervised members for projects, internships, and/or theses on a topic of interest in an official capacity.

Students, faculty, and researchers from other institutions are eligible for honorary membership with the approval of any officer. Past members are eligible for honorary membership with the approval of any officer. Such memberships may be revoked by an executive decision of the officers. Such a revocation is permanent.

\section{OFFICERS}
§ 1. The positions of the officers in this organization shall be, in order of rank, president and treasurer.

The duties of the officers shall be to attend executive board meetings and make decisions about the regular functioning of the organization. Additionally, the president shall be responsible for all internal and external correspondence of the organization as well as any necessary record-keeping related to the membership of the organization and serve as the liaison between the organization and Campus Life. The treasurer shall be responsible for all financial matters of the organization, including any necessary record-keeping related to such matters.

§ 2. Only voting members, who are students, are qualified to be officers. Officers are required to abide by the Student Code of Conduct as defined in the RIT Governance Policy §D18.0.IV.

§ 3. Each officer shall have one vote, and decisions shall be made by unanimity.

\section{ELECTION \& IMPEACHMENT}
§ 1. Officers shall be elected to one-year terms starting and ending at the close of an academic year. Elections are to take place during the third- or second-to-last week of the academic year to allow time for leadership transition.

§ 2. Voting is to be conducted by voting members of the organization, with each voting member having one vote. In the case of multiple people running for a position, ranked-choice voting shall be used for the election to that position.

§ 3. If an officer no longer fulfills the qualification requirements, they shall be removed from office, and an election to replace them shall be held at the next meeting (as described in ARTICLE VI). Other officers shall immediately notify Campus Life of such an event.

If an officer resigns before the conclusion of their term, they shall inform the other officers, and an election to replace them shall be held at the next meeting (as described in ARTICLE VI).

\section{ADVISOR}
§ 1. The advisor shall be appointed by the officers through executive decision. The advisor must be a mentor.

If the advisor stops being an employee of RIT, they shall be relieved of their duties. A new advisor must be appointed at the next executive board meeting.

If the advisor resigns, they shall inform the board members, and a new advisor must be appointed at the next executive board meeting.

§ 2. The conduct of the advisor will be dictated by the Advisor Handbook. In particular, the advisor shall be responsible for assisting the president in their role as the liaison between the organization and Campus Life and assisting the officers in reaching unanimity in their executive decisions.

\section{MEETINGS}
§ 1. Hour-long meetings are to be held weekly during the semesters. Meetings will involve events based on prior communication by the secretary. Any pending officer elections must be held before the start of the event. After any such elections, the president may propose constitutional amendments as described in ARTICLE VIII. Before the start of the event, but after any elections and/or constitutional amendment proposals, any voting member may propose additional events for the consideration of the officers.

§ 2. Additional events proposed during the regular meetings are special events, the logistics of which shall be decided through executive decision. Executive board meetings are special events, the logistics of which shall be decided by the officers at their own discretion. All other events are non-special events.

§ 3. Attendance at events shall be tracked through CampusGroups.

§ 4. A quorum shall consist of at least 75\% of the voting members. Such a quorum is necessary for officer elections and constitutional amendments.

§ 5. Officers shall be elected as per ARTICLE IV. Constitutional amendments shall be decided on as per ARTICLE VIII. Executive decisions shall govern all other organizational activities.

§ 6. Robert’s Rules of Order shall be used whenever necessary during the meetings and the special events.

\section{COMMITTEES}
None

\section{AMENDMENTS AND BY-LAWS}
§ 1. Officers may propose and discuss amendments to the constitution at executive board meetings.

§ 2. The amended constitution must be proposed to the general membership at the next meeting (described in ARTICLE VI).

§ 3. Amendments shall be ratified by a majority vote of the quorum present at the meeting where the amendment is proposed.

§ 4. The constitution shall serve as the only governing document of the organization. Governance and procedural details shall be established only through the amendment process; no separate bylaws shall be recognized.

\section{ADHERENCE TO UNIVERSITY POLICIES}
§ 1: Anti-Hazing

Per the RIT Hazing Policy (RIT Student Conduct Process; IV. RIT Code of Conduct; 14. Hazing/Failure to Report Hazing)

Hazing/Failure to Report Hazing. Behavior, regardless of intent, which endangers the emotional, or physical health and safety of a Student for the purpose of membership, affiliation with, or maintaining membership in, a group or Student Organization. Hazing includes any level of participation, such as being in the presence, having awareness of hazing, or failing to report hazing. Examples of hazing include, but are not limited to, beating or branding, sleep deprivation or causing excessive fatigue, threats of harm, forcing or coercing consumption of food, water, alcohol or other drugs, or other substances, verbal abuse, embarrassing, humiliating, or degrading acts, or activities that induce, cause or require the Student to perform a duty or task which is not consistent with fraternal law, ritual or policy or involves a violation of local, state or federal laws, or the RIT Code of Conduct.

NY State Hazing Law

§ 120.16 Hazing in the first degree. A person is guilty of hazing in the first degree when, in the course of another person's initiation into or affiliation with any organization, he intentionally or recklessly engages in conduct which creates a substantial risk of physical injury to such other person or a third person and thereby causes such injury. Hazing in the first degree is a class A misdemeanor.

§ 120.17 Hazing in the second degree. A person is guilty of hazing in the second degree when, in the course of another person's initiation or affiliation with any organization, he intentionally or recklessly engages in conduct which creates a substantial risk of physical injury to such other person or a third person. Hazing in the second degree is a violation.

§ 2: Anti-Discrimination Clause

This organization shall not discriminate on the basis of sex, race, color, sexual orientation gender identity and gender expression, religion, age marital sate, national origin, disability or veteran status. This policy will include but is not limited to, recruiting, membership, organization activities or opportunities to hold or run for club office.

§ 3: Statement of Compliance with University regulations

This organization shall comply with all university and Center for Campus Life policies and regulations, and local, state and federal laws.

\section*{ENABLING CLAUSE}
This constitution was voted on and put into effect on Jan 22nd, 2026.

Signature line (include date):

President:Date:

		 (Chandradeep Dey)
         
Treasurer:Date:

		 (Jeremy Stratton-Smith)
         
Advisor:Date:

		 (Arthur Azevedo de Amorim)

\end{document}
